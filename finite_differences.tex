\documentclass{article}

\usepackage{amsmath, amsthm, bm, physics, enumerate, siunitx, dirtytalk, tikz}
\usepackage{hyperref}
\usetikzlibrary{calc, patterns}

\newtheorem{theorem}{Theorem}
% \newtheorem{lemma}{Lemma}

% \providecommand{\levicivita}[3]{%
%   \ensuremath{\varepsilon_{{#1}{#2}{#3}}}
% }


% ------------------------------------------------------------------------------
\title{Formulas for finite differences}
\author{Wojciech Sadowski}

\begin{document}
\maketitle
\section{Basics}
\paragraph{Taylor series}
\[
  u(x + h) = u(x) 
  + \frac{h}{1!}\pdv{u}{x} 
  + \frac{h}{2!}\pdv[2]{u}{x} 
  + \frac{h}{3!}\pdv[3]{u}{x} + \dots
\]

\section{One dimension}
\paragraph{Forward difference}
\[
  u(x + h) = u(x) 
  + \frac{h}{1!}\pdv{u}{x} 
  + \frac{h^2}{2!}\pdv[2]{u}{x} +
  + \frac{h^3}{3!}\pdv[3]{u}{x} +
  \mathcal{O}\qty(h^4)
\]
We divide by \(h\):
\[
  \frac{u(x + h)}{h} = \frac{u(x)}{h} 
  + \pdv{u}{x} 
  + \frac{h}{2!}\pdv[2]{u}{x} 
  + \frac{h^2}{3!}\pdv[3]{u}{x} 
  + \mathcal{O}\qty(h^3)
\]
Rearange:
\begin{equation}\label{eq::forward}
   \pdv{u}{x} 
   = \frac{u(x + h) - u(x)}{h}  
  - \frac{h}{2!}\pdv[2]{u}{x} 
  - \frac{h^2}{3!}\pdv[3]{u}{x} 
  + \mathcal{O}\qty(h^3)
\end{equation}
Finally:
\[
   \pdv{u}{x} 
   = \frac{u(x + h) - u(x)}{h}  + \mathcal{O}\qty(h)
\]

\paragraph{Backward difference}
\[
  u(x - h) = u(x) 
  - \frac{h}{1!}\pdv{u}{x} 
  + \frac{h^2}{2!}\pdv[2]{u}{x} 
  - \frac{h^3}{3!}\pdv[3]{u}{x} 
  + \mathcal{O}\qty(h^4)
\]
We divide by \(h\):
\[
  \frac{u(x - h)}{h} = \frac{u(x)}{h} 
  -\pdv{u}{x} 
  + \frac{h}{2!}\pdv[2]{u}{x} +
  - \frac{h^2}{3!}\pdv[3]{u}{x} +
  \mathcal{O}\qty(h^3)
\]
Rearange:
\begin{equation}\label{eq::backward}
   \pdv{u}{x} 
   = \frac{u(x) - u(x-h)}{h}  
  + \frac{h}{2!}\pdv[2]{u}{x} 
  - \frac{h^2}{3!}\pdv[3]{u}{x} 
  + \mathcal{O}\qty(h^3)
\end{equation}
Finally:
\[
   \pdv{u}{x} 
   = \frac{u(x) - u(x - h)}{h}  + \mathcal{O}\qty(h)
\]
\paragraph{Central difference}
Take a sum of forward and backward, \autoref{eq::forward} and 
\autoref{eq::backward}:

\begin{equation}\label{eq::backward}
   2\pdv{u}{x} 
   = \frac{u(x+h) - u(x-h)}{h}  
  - \frac{2 h^2}{3!}\pdv[3]{u}{x} 
  + \mathcal{O}\qty(h^3)
\end{equation}
Divide by two:
\begin{equation}\label{eq::backward}
   \pdv{u}{x} 
   = \frac{u(x+h) - u(x-h)}{2h}  
  - \frac{1 h^2}{3!}\pdv[3]{u}{x} 
  + \mathcal{O}\qty(h^3)
\end{equation}
Finally:
\begin{equation}\label{eq::backward}
   \pdv{u}{x} 
   = \frac{u(x+h) - u(x-h)}{2h}  
  + \mathcal{O}\qty(h^2)
\end{equation}

\paragraph{Central difference of second derivative}
It is not exactly central difference of a central difference. That would have 
a wide stencil. Bad. Instead try writing at halfs.
\[
  u''(x) = \frac{u'(x+h/2) - u'(x-h/2)}{h}
\]
\[
  u'(x\pm h) = \frac{u(x\pm h/2 + h/2) - u(x \pm h/2 - h/2)}{h}
\]
Pluging in\dots
\[
  u''(x) = \frac{ u(x + h) - 2 u(x) + u(x-h) }{h^2}
\]

\section{Solving systems of equations}
\paragraph{Gauss seidel formula}
From Wikipedia. We have 
\[
  A =
\underbrace{
\begin{bmatrix} a_{11} & 0 & \cdots & 0 \\ a_{21} & a_{22} & \cdots & 0 \\ \vdots & \vdots & \ddots & \vdots \\a_{n1} & a_{n2} & \cdots & a_{nn} \end{bmatrix}
}_{\textstyle L_*}
+
\underbrace{
\begin{bmatrix} 0 & a_{12} & \cdots & a_{1n} \\ 0 & 0 & \cdots & a_{2n} \\ \vdots & \vdots & \ddots & \vdots \\0 & 0 & \cdots & 0 \end{bmatrix}
}_{\textstyle U}
.
\]
We do:
\begin{alignat}{1}
A\mathbf x &= \mathbf b \\
(L_*+U) \mathbf x &= \mathbf b \\
L_* \mathbf x+U\mathbf x &= \mathbf b \\
L_* \mathbf{x} &= \mathbf{b} - U \mathbf{x}
\end{alignat}

The Gauss–Seidel method now solves the left hand side of this expression for $\mathbf{x}$, 
using previous value for $\mathbf{x}$ on the right hand side. Analytically, 
this may be written as:
\[
  \mathbf{x}^{(k+1)} = L_*^{-1} \left(\mathbf{b} - U \mathbf{x}^{(k)}\right).
\]

However, by taking advantage of the triangular form of \(L_{*}\), the elements of 
\(\mathbf {x} ^{(k+1)}\) can be computed sequentially for each row \(i\) using 
forward substitution:
\[
  x^{(k+1)}_i  = \frac{1}{a_{ii}} \left(b_i - \sum_{j=1}^{i-1}a_{ij}x^{(k+1)}_j - \sum_{j=i+1}^{n}a_{ij}x^{(k)}_j \right),\quad i=1,2,\dots,n
\]

\end{document} 
