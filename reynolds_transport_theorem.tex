\documentclass{article}

\usepackage{amsmath, amsthm, bm, physics, enumerate}
\usepackage{hyperref}
\newtheorem{theorem}{Theorem}
\newtheorem{lemma}{Lemma}

\providecommand{\levicivita}[3]{%
  \ensuremath{\varepsilon_{{#1}{#2}{#3}}}
}

% ------------------------------------------------------------------------------
\title{Derivation of Reynolds transport theorem}
\author{Wojciech Sadowski}

\begin{document}
\maketitle

\begin{theorem}[Reynolds transport theorem, RTT]
  Given:
  \begin{enumerate}[(i)]
    \item a material volume \(\Omega\), whose shape and position are varying with time,
      i.e.\ \(\Omega = \Omega(t)\);
    \item an extensive variable \(\Phi = \Phi(t)\) with its intensive counterpart 
      \(\phi = \phi(t, \vb{x})\)  defined inside the material volume.
  \end{enumerate}
	The total derivative of \(\Phi\) with respect to time can be formulated as:
  \begin{equation}
    \dv{\Phi}{t} = \dv{t} \int\limits_{\Omega(t)} \phi \dd{\Omega}
    = \int\limits_{\Omega(t)} \dfrac{D\phi}{Dt} + \phi \div{\vb{v}} \dd{\Omega}
  \end{equation}
\end{theorem}

\begin{proof}
	The integral of \(\phi\) can be expressed in reference configuration 
  \(\Omega^0\), assuming that \(\Omega(t) = J(t)\Omega^0\):
	\begin{equation}
    \dv{t} \int\limits_{\Omega(t)} \phi(t, \vb{x}) \dd{\Omega} = 
    \dv{t} \int\limits_{\Omega^0} \phi(t, \vb{x^0})J(t) \dd{\Omega^0} 
	\end{equation}
	Since \(\Omega_0\) is time independent, the derivative can be moved
  under the integral:
	\begin{equation}
    \dv{t} \int\limits_{\Omega(t)} \phi(t, \vb{x}) \dd{\Omega} = 
    \int\limits_{\Omega^0} \dv{t}[\phi(t, \vb{x^0})J(t)] \dd{\Omega^0} =
    \int\limits_{\Omega^0} \dfrac{D\phi}{Dt}J + \phi\dv{J}{t} \dd{\Omega^0} 
	\end{equation}
  Note the presence of substantial derivative; the intensive quantity is
  transported by the fluid.
	Factoring the Jacobian out of the terms under the integral, we can return
  to the original integration region
	\begin{equation}
    \int\limits_{\Omega^0} \dfrac{D\phi}{Dt}J + \phi\dv{J}{t} \dd{\Omega^0} = 
    \int\limits_{\Omega^0} \qty[\dfrac{D\phi}{Dt} + J^{-1}\phi\dv{J}{t}]J \dd{\Omega^0} = 
    \int\limits_{\Omega(t)} \dfrac{D\phi}{Dt} + J^{-1}\phi\dv{J}{t} \dd{\Omega}.
	\end{equation}
  The final formula is obtained by plugging in the time derivative of the 
  Jacobian~\autoref{eq::ddt_J}:
  \begin{equation*}
    \int\limits_{\Omega(t)} \dfrac{D\phi}{Dt} + J^{-1}\phi\dv{J}{t} \dd{\Omega}
    = \int\limits_{\Omega(t)} \dfrac{D\phi}{Dt} + \phi \div\vb{v} \dd{\Omega}
  \end{equation*}

  
\end{proof}



\section*{Auxiliary definitions and lemmas}

\paragraph{Material volume}
Special integration region \(\Omega\) defined as a parcel of fluid
with such boundaries that there is no flux of selected intensive property 
through the them.

\paragraph{Index notation and Levi-Civita symbol}
Index notation, or Einstein notation simplifies greatly multi variable 
calculus and relies on two rules:
\begin{enumerate}
  \item Any vector variable can be represented as symbol with an index.
    For example, velocity \(\vb{v}\), a vector having generally three 
    components, can be simply written as \(v_i\). Index \(i\) is the so 
    called \textbf{free index}. The letter does not matter, i.e.\ \(i\) can
    be replaced with \(\alpha\) or whatever.
  \item Repeated index implies summation over full range of this index. 
    This means that, if a term like \(a_i b_j c_j\) is encountered, it can
    be expanded into
    \[
      a_i b_j c_j = a_i b_1 c_1 + a_i b_2 c_2 + a_i b_3 c_3 =
      \sum\limits_{j=1,2,3} a_i b_j c_j.
    \]
    Therefore, we can say that the sum sign \(\Sigma\) is "implicit". The index
    \(i\) remained a free index in the above expression. This means that a
    version for each spatial dimension can be obtained by setting \(i\) to 
    a chosen index.
\end{enumerate}
The Levi-Civita symbol, \(\levicivita{i}{j}{k}\), is a helpful permutation
symbol defined in a following way:
\begin{equation*}
\varepsilon_{ijk} = \begin{cases}
         +1 & \text{if } (i,j,k) \text{ is } (1,2,3), (2,3,1), \text{ or } (3,1,2), \\
         -1 & \text{if } (i,j,k) \text{ is } (3,2,1), (1,3,2), \text{ or } (2,1,3), \\
    \;\;\,0 & \text{if } i = j, \text{ or } j = k, \text{ or } k = i
\end{cases}
\end{equation*}

\paragraph{Transforming material volume to reference configuration}
The integration region is usually the function of time, \(\Omega = \Omega(t)\).
The reference configuration of \(\Omega(t)\), denoted as \(\Omega^0\), is not 
time independent. Assuming that transformation \(\Omega^0 \rightarrow \Omega\)
is invertable, we can change the integration region to the reference
configuration by writing:
\begin{equation}
  d\Omega(t) = J(t)d\Omega^0,
  \label{eq::jacobian_relation}
\end{equation}
where \(J\) denotes a Jacobian of the transformation.

\paragraph{Defining the Jacobian}
The trajectory of a point \(\vb{x}\) laying inside 
the parcel \(\Omega\) is a function of time and the initial 
configuration \(\vb{x}^0\):
\begin{equation}
  \vb{x} = \vb{x}(t, \vb{x}^0)
  \label{eq::trajectory}
\end{equation}
In the above, we have assumed that for a given point it is possible 
to find its reference configuration, that is inverse function 
\(\vb{x}^0 = \vb{x}^0(t, \vb{x})\) exists.
The necessary and sufficient condition for invertability is the 
non-vanishing Jacobian, which can be given as:
\begin{equation}\label{eq::jacobian}
  J = \det\pdv{x_i}{x^0_j} 
  = \levicivita{i}{j}{k} \pdv{x_1}{x^0_i}\pdv{x_2}{x^0_j}\pdv{x_3}{x^0_k}
\end{equation}

\paragraph{Time derivative of intensive variable, substantial derivative}
Total time derivative of an intensive property \(d \phi / d t\) 
(also called substantial derivative, denoted \(D\phi/Dt\)) can be formulated 
in the following way. We assume that trajectory of each point inside a fluid can
be described as in \autoref{eq::trajectory} and 
we use the chain rule:
\begin{equation*}
  \dfrac{D\phi}{Dt} \equiv \dv{\phi}{t}  
  =  \dv{t}f\qty(\vb{x}\qty(t,\vb{x}^0))
  = \pdv{\phi}{t} + \dv{x_i}{t} \pdv{\phi}{x_i}
  = \pdv{\phi}{t} + \qty(\dv{\vb{x}}{t} \dotproduct \grad )\phi
\end{equation*}
Since the time derivative of the trajectory \(d\vb{x} / dt\) is equal 
to the fluids velocity \(\vb{v}\) by definition, the above reduces to:
\begin{equation}
  \dfrac{D\phi}{Dt} \equiv \dv{\phi}{t} 
  = \pdv{\phi}{t} + (\vb{v} \dotproduct \grad) \phi
\end{equation}


\paragraph{Material derivative of the Jacobian}
The Jacobian arises most often while changing the integration region 
from \(\Omega(t)\) to \(\Omega^0\) according to \autoref{eq::jacobian_relation}.
The closed expression for the time derivative of the Jacobian must be found 
for the proof or RTT.
Starting from the definition of the Jacobian determinant~\autoref{eq::jacobian},
and distributing the derivative:
\begin{align*}
  \dv{J}{t} = \dv{t} \qty(\levicivita{i}{j}{k} 
  \pdv{x_1}{x^0_i} \pdv{x_2}{x^0_j} \pdv{x_3}{x^0_k})
  &= \levicivita{i}{j}{k} \dv{t} \qty( \pdv{x_1}{x^0_i} ) 
    \pdv{x_2}{x^0_j} \pdv{x_3}{x^0_k} \\
  &+ \levicivita{i}{j}{k} \pdv{x_1}{x^0_i} 
    \dv{t} \qty( \pdv{x_2}{x^0_j})  \pdv{x_3}{x^0_k} 
  + \levicivita{i}{j}{k} \pdv{x_1}{x^0_i} 
    \pdv{x_2}{x^0_j} \dv{t} \qty( \pdv{x_3}{x^0_k} )  
\end{align*}
In the above, differentiation with respect to time and reference
configuration coordinates commute (the order of differentiation 
can be freely changed), resulting in velocity components:
\begin{equation*}
  \dv{t} \qty( \pdv{x_\alpha}{x^0_\beta} )
  = \pdv{v_\alpha}{x^0_\beta}
  = \pdv{v_\alpha}{x_\gamma} \pdv{x_\gamma}{x^0_\beta}
\end{equation*}
We also insert additional differentiation for the sake of 
further derivation.

By inserting the above into the formula for the Jacobian derivative 
and simplifying, we get final formula for the \(\diffd J/\diffd t\).
Taking only one of the terms as an example and expanding summation 
over \(\gamma\):
\begin{equation*}
  \begin{split}
    & \levicivita{i}{j}{k} \dv{t} \qty( \pdv{x_1}{x^0_i} ) 
      \pdv{x_2}{x^0_j} \pdv{x_3}{x^0_k}
    = \pdv{v_1}{x_\gamma} \levicivita{i}{j}{k} \pdv{x_\gamma}{x^0_i}\pdv{x_2}{x^0_j}
      \pdv{x_3}{x^0_k} =\\
    & \pdv{v_1}{x_1} \levicivita{i}{j}{k} \pdv{x_1}{x^0_i}\pdv{x_2}{x^0_j} \pdv{x_3}{x^0_k} + 
    \pdv{v_1}{x_2} \levicivita{i}{j}{k} \pdv{x_2}{x^0_i}\pdv{x_2}{x^0_j} \pdv{x_3}{x^0_k} +
    \pdv{v_1}{x_3} \levicivita{i}{j}{k} \pdv{x_3}{x^0_i}\pdv{x_2}{x^0_j} \pdv{x_3}{x^0_k}
  \end{split}
\end{equation*}
Both terms on the right with repeated index in numerators will be equal to 
zero because of the definition of \(\levicivita{i}{j}{k}\). They will be 
repeated two times, once positive and once negative, for example:
\[
  \pdv{v_1}{x_2} \levicivita{i}{j}{k} 
  \pdv{x_2}{x^0_i}\pdv{x_2}{x^0_j} \pdv{x_3}{x^0_k} = \dots
  \pdv{v_1}{x_2} 
  \pdv{x_2}{x^0_1}\pdv{x_2}{x^0_2} \pdv{x_3}{x^0_3} + \dots 
  - \pdv{v_1}{x_2} 
  \pdv{x_2}{x^0_2}\pdv{x_2}{x^0_1} \pdv{x_3}{x^0_3} + \dots
\]
The terms on the right cancel each other out. In the above, we shown only 
\((i,j,k) = (1,2,3)\) and \((2,1,3)\).
Thanks to that, and using the original Jacobian definition, 
i.e.~\autoref{eq::jacobian}, we can simplify first of the terms: 
\begin{equation*}
  \levicivita{i}{j}{k} \dv{t} \qty( \pdv{x_1}{x^0_i} ) \pdv{x_2}{x^0_j} 
  \pdv{x_3}{x^0_k}
  = \pdv{v_1}{x_1} \levicivita{i}{j}{k} \pdv{x_1}{x^0_i}\pdv{x_2}{x^0_j} 
  \pdv{x_3}{x^0_k}
  = \pdv{v_1}{x_1} J
\end{equation*}
Treating indices \(2\) and \(3\) the same way, the final formula for the 
time derivative of the Jacobian looks as follows:
\begin{equation}
  \dv{J}{t}  = \pdv{v_i}{x_i} J = \qty(\div\vb{v})J
  \label{eq::ddt_J}
\end{equation}







% \begin{proof}
% 	The integral of \(f\) can be expressed in reference configuration \(\Omega^0\) using the relation~\eqref{eq::theo::reynolds::regions}:
% 	\begin{equation}
% 		\dfrac{d}{dt}\regionint{\Omega(t)}{f}{\Omega}
% 		= \dfrac{d}{dt}\regionint{\Omega^0}{f(t, \vec{x}(t, \vec{x}^0)) J}{}
% 	\end{equation}
% 	Since \(\Omega_0\) does not change in time the derivative can be moved under the integral:
% 	\begin{equation}
% 		\dfrac{d}{dt}\regionint{\Omega(t)}{f(t, \vec{r})}{\Omega}
% 		=
% 		\regionint{\Omega^0}{\dfrac{d}{dt} \left\{ f(t, \vec{x}(t, \vec{x}^0)) J \right\}}{\Omega^0}
% 		=\regionint{\Omega^0}{\dfrac{df}{dt}J + f\dfrac{dJ}{dt}}{\Omega^0}
% 	\end{equation}
% 	Dividing what is under the integral by the Jacobian, we can return to the original integration region:
% 	\begin{equation}
% 		\regionint{\Omega^0}{\dfrac{df}{dt}J + f\dfrac{dJ}{dt}}{}
% 		= \regionint{\Omega^0}{\left[ \dfrac{df}{dt} + \dfrac{1}{J}f\dfrac{dJ}{dt} \right]J}{}
% 		= \regionint{\Omega(t)}{\dfrac{df}{dt} + f\dfrac{1}{J}\dfrac{dJ}{dt}}{\Omega}
% 	\end{equation}
% 	The final formula is obtained by plugging in the time derivative of the Jacobian~\eqref{eq::theo::reynolds::ddt(J)}:
% 	\begin{equation*}
% 		\dfrac{d}{dt}\regionint{\Omega(t)}{f}{\Omega}
% 		= \regionint{\Omega(t)}{\dfrac{df}{dt} + f\dfrac{1}{J}\dfrac{dJ}{dt}}{\Omega}
% 		= \regionint{\Omega(t)}{ \dfrac{df}{dt} + f (\nabla \cdot \vec{v})}{\Omega}
% 	\end{equation*}
% \end{proof}

\end{document}
