\documentclass[justified, nobib]{tufte-handout}

\title{Heavily annotated derivation of the conservation laws of fluid 
  mechanics\thanks{%
    Greatly inspired by the lectures of prof. Andrzej Styczek, 
    scientist and teacher at Faculty of Power and Aeronautical Engineering of
    Warsaw University of Technology.}}

\author[Wojciech Sadowski]{Wojciech Sadowski}

%\date{28 March 2010} % without \date command, current date is supplied

% \geometry{showframe} % display margins for debugging page layout

% setting ptions for included pictures
\usepackage{graphicx} % allow embedded images
\graphicspath{{graphics/}} % set of paths to search for images
\setkeys{Gin}{
  width=\linewidth,
  totalheight=\textheight,
  keepaspectratio
}

\hypersetup{colorlinks}

\usepackage[compatibility=false, font=footnotesize]{caption}
% * options are necessary to fix the bug that autoref was causing,
%   figure was constantly changed to section
\DeclareCaptionFont{blue}{\color{blue}}
\captionsetup{labelfont={blue}}

% color and color palettes
\usepackage{xcolor}

% drawings collor palette
\definecolor{p11}{HTML}{33cfff}
\definecolor{p12}{HTML}{ffa733}
\definecolor{p13}{HTML}{ff4133}



% drawing inside tex
\usepackage{tikz} 

% math things
\usepackage{
  amsmath,  % extended mathematics
  siunitx,  % non-stacked fractions and better unit spacing
  bm,       % bold math symbols
  physics,  % a set of usefull symbols for differentiation 
  theorem   % provides theorem env
}

% table things
\usepackage{
  booktabs, % book-quality tables
  multicol  % multiple column layout facilities
} 

% quoting text and friends
\usepackage{fancyvrb}         % extended verbatim environments
\fvset{fontsize=\normalsize}  % default font size for fancy-verbatim environments
\usepackage{dirtytalk}        % provides \ay command for easy quiting 

% citing sources
\usepackage{natbib}
\setcitestyle{authoryear}
\bibliographystyle{plainnat}

% misc 
\usepackage{lipsum}   % filler text


\newcommand{\wikilink}[2][Wikipedia]{%
  \href{https://en.wikipedia.org/wiki/#2}{#1}%
}
\newcommand{\wikicite}[1]{%
  \textsuperscript{\wikilink[W]{#1}}
}

\renewcommand{\vec}[1]{\bm{#1}}

% definition necessary for the tensor command
% -------------------------------------------
\DeclareMathAlphabet{\mathsfbi}{OT1}{\sfdefault}{bx}{sl}
\DeclareMathVersion{sfletters}
\SetSymbolFont{letters}{sfletters}{OML}{ntxsfmi}{b}{it}

\makeatletter
\newcommand{\mathbfsbilow}[1]{%
  \text{\mathversion{sfletters}$\m@th#1$}%
}

\DeclareRobustCommand{\tnsr}[1]{%
  \begingroup
  \ifcat\noexpand #1\relax
    % assume Greek letter
    \edef\greek@test{\detokenize{#1}}%
    \edef\greek@test{\expandafter\@cdr\greek@test\@nil}%
    \edef\greek@test{\expandafter\@car\greek@test\@nil}%
    \edef\x{\the\lccode\expandafter`\greek@test}%
    \edef\y{\number\expandafter`\greek@test}%
    \ifnum\x=\y\relax
      % the command name starts with a lowercase letter
      \mathbfsbilow{#1}%
    \else
      \mathsfbi{#1}%
    \fi
  \else
    \mathsfbi{#1}%
  \fi
  \endgroup
}
\makeatother
% -------------------------------------------

\newcommand{\name}[1]{%
  \textbf{#1}%
}

\newtheorem{theorem}{Theorem}

\newcolumntype{L}[2]{>{\raggedleft#2}p{#1\textwidth}}

% symbols macros
\def\stress{T}
\def\stressT{\tnsr{\stress}}
\def\strainT{\tnsr{D}}



\begin{document}

\maketitle% this prints the handout title, author, and date

\begin{abstract}
\noindent
  \lipsum[1]{}
\end{abstract}

\section{The basics}

\subsection{What is a fluid?}
\textsc{We will consider a fluid as a continuous substance,} which fills fully
any available space. We therefore are effectively \emph{neglecting the discrete
nature of matter.} In simple terms, since the amount of fluid molecules in a
small volume of matter is obscenely large (remember the Avogadro number%
\footnote{
  The \wikilink[Avogadro constant]{Avogadro\_constant} denotes a number of
  particles of a substance per a \si{mol} of that substance. It is equal to
  \SI{6.02214076e23}{mol^{-1}}. For reference, one \si{mol} of an ideal gas
  under one atmosphere of pressure and at \SI{25}{\degree C} fits in a
  approximately \SI{30}{cm} cube.\wikicite{Molar\_volume\#Ideal\_gases}
} 
\(N_\mathrm{A}\approx\SI{6.022e23}{mol^{-1}}\)) this is a fair assumption. We
will provide better, more mathematical justification for this modelling
approach shortly, but for now, let's discuss why the opposite, i.e.\ directly
treating the fluid as a \emph{set of molecules,} is cumbersome.

The intuitive idea that we could obtain macroscopic fluid parameters, like
density or temperature, from the information about the states of all the
particles is correct. Unfortunately, it is also a rather unpractical idea, as it
is painfully difficult to describe the motion of each and every molecule
constituting the substance, i.e.\ gather the required information. There are
two main difficulties for such process. Apart from the obvious problem of
computationally expensive integration all of the trajectories,%
\footnote{%
  The path of motion \(\vec{x}(t)\) and velocity \(\vec{v}(t)\) of a point
  particle with mass \(m\) can be computed by integrating Newton's second law
  of motion
  \[
    m\dv[2]{\vec{x}}{t} = \vec{F},
  \]
  assuming that \(\vec{F}(t, \vec{x})\) is the force acting on the particle.
}
we also encounter a tricky puzzle of determining an accurate initial conditions
for every molecule. In such systems, a small error in determining these
starting parameters will unfortunately lead to considerable inaccuracies
further down the line. Hence, we neglect the discrete nature of matter and we
imagine that the medium, i.e.\ the fluid, consists of \emph{continuous set} of ideal
particles%
\footnote{%
	A particle that might have mass, charge or some other property, however it does
	not have any dimension, volume or as nicely stated in
	\wikilink{Point\_particle}, it does not occupy space.
} 
(think about in the same way as about infinite set of real numbers).

Before we move on, we still need to define when it is a good idea to use the
\emph{continuum mechanics} formulation of the fluid flow. Here, we stumble upon
an important physics concept: \emph{modelling}. The assumption of continuity of
the fluid is to some extent an approximation of reality, which can be very
useful to describe what we actually observe. The reality is naturally very
complex and we all know that \emph{less is more:} eliminating unnecessary
details often leads to clearer picture. However sometimes, it might be
difficult to \say{guesstimate} if we are not committing a grave error by using
a modelling approach, which simplifies reality too much. So it is undeniably
useful to have a mathematical indicator of how \say{continuous} given problem
is. 

Fortunately, we do have a device like that! We can compute a ratio of the 
molecule mean free path \(\lambda\) to the characteristic dimension of the 
flow \(L\). The first one describes the distance covered by a particle before 
it changes direction, usually due to a collision. This fraction is the definition
of the \wikilink[Knudsen number]{Knudsen\_number}
\[
  \mathrm{Kn} = \dfrac{\lambda}{L},
\]
which differentiates flow regimes~\citep{collin2014}. As \(\mathrm{Kn}\)
approaches and climbs over \num{e-3}, more and more effects related to the
\say{granularity} of matter have to be included (see
\autoref{tab:kn-regiemes}). For \(\mathrm{Kn} \ll 1\) we can assume continuity
of the fluid, which we will do for the remainder of this text. 

\begin{margintable}
  \caption{Flow regimes as a function of \(\mathrm{Kn}\) \citep{collin2014}.
  \emph{Continuum flow} is described by Navier-Stokes equations. In \emph{slip
  flow} they are still applicable, however special treatment of walls is
  necessary due to molecular effects. The equations became invalid in
  \emph{transitional flows}, however the effect of collisions between molecules
  is still significant. Finally, in \emph{free molecular flow} this phenomenon
  can also be neglected.}
  \label{tab:kn-regiemes}
  \centering
  \begin{tabular}{ll}
  \toprule
    \(\mathrm{Kn}\) range & Flow type \\ \midrule
    \(\mathrm{Kn} < \num{e-3}\)                 & continuum\\
    \( \num{e-3} < \mathrm{Kn} < \num{e-1}\) & slip\\
    \( \num{e-1} < \mathrm{Kn} < \num{10}\)  & transitional\\
    \( \mathrm{Kn} < \num{10}\)              & free molecular\\
  \bottomrule
  \end{tabular}
\end{margintable}

\newthought{Circling back} to our deliberations about how to describe the
movement of the fluid, we can \emph{de facto} equate it to the movement of
these material points. As it turns out, such a description can be used for
fluids, solids and \say{things} in between. In comparison to solids, which have
some \say{rigid} reference configuration or shape, fluids do not exhibit this
property. The relative positions of the points can be arbitrary and changing.
Naturally, while approaching a fluid mechanics problem, we have to make certain
assumptions about this movement, otherwise deducing anything about the problem
would be impossible. For example, sometimes we might assume that volume of
fluid has to be conserved (or that the density is constant) but that is, of
coarse, a simplification. A useful and accurate one in situations of
non-varying thermodynamic state of the medium.

If this \say{reference shape} does not exist, than the deformation (or strain)
of the fluid can be completely arbitrary. Perhaps at first this thought feels
unintuitive, but if you consider the water spilled on the floor and left to its
own devices, it will deform greatly, covering a big area and even seep into the
cracks, potentially making you call the handyman to fix the broken flooring.
Yet, there has to be some set of internal forces that govern the flow. We will
assume that they do not depend on the deformation, but we still have to answer
the question: what generates these forces inside the fluid? One obvious answer
is the change of volume, i.e.\ compression, but what else?

\newthought{We will consider the fluids} in which the forces depend on the rate
of deformation. These forces clearly will \say{want} to stop, or counteract the
deformation occurring in the fluid. As we have already discussed, the
deformation itself does not seem to be significant. However, it is important to
note, that this is only a simplified model, not truly existing in nature. In
the same way as the assumption of incompressibility, which is never
\emph{really} true, as each fluid is compressible, it is a useful lie that does
a good job describing the reality as we can measure it.%
\footnote{This assumption does not cover every \say{flowing} substance. For
example, \wikilink[Bingham plastic]{Bingham\_plastic} resists the deformation
until yield stress value and only then starts flowing. Mud, slurry, toothpaste
are real examples of such fluids.}
For the most general fluid we could write
\[
	Z(t, \stressT, \dot{\stressT}, \tnsr{D}, \dot{\tnsr{D}}, \dots) = 0,
\]
and call the above a \emph{constitutive equation}, i.e.\ the relation defining
our material. It describes connection between the time \(t\), stresses
\(\stressT\), deformation \(\strainT\), the rates of those things (indicated by
the same letters with a dot accent) and potentially other parameters.

As you can see, it can be bad. The parameters in the above relations could
depend on each other or even they could depend on each other non-locally!%
\footnote{%
	For example, stress in point \(a\) depends on the state in point \(b\).
}
Or even worse, they might be influenced by the history of the those parameters.%
\footnote{%
	That would be non-locality in time, if you want to see an example of such
	\say{difficult} relations in hydrodynamics you can read about
	\wikilink[Basset force]{Basset\_force} or \wikilink[BBO
		equation]{Basset-Boussinesq-Oseen\_equation}.
}
Fortunately, for most of classical use cases we can confine ourselves to much
simpler relations. Only the following things will be considered:
\begin{enumerate}[(i)]
	\item only \(\stressT\) and \(\dot{\strainT}\) are important;
  \item the relation between them is local, i.e.\ it is depend on the values in
    the same place and time;
  \item we should be able two solve for \(\stressT\), that is write 
    \(\stressT = g(\dot{\strainT})\) where \(g\) is some function with tensor argument;
	\item the relationship should not have any preferential direction. 
\end{enumerate}
To illustrate better the last point, the fluid, and hence the constitutive
relation, should not behave differently in different coordinate system. Or to
frame it as a simple example, considering a coordinate system with axes \(x\),
\(y\) and \(z\), the fluid flowing along the \(x\) direction should not behave
differently than the same fluid moving along \(y\) axis.%
\footnote{Perhaps we are getting ahead of ourselves, but in practice, this will
require \(\stressT\) and \(\dot{\strainT}\) to be symmetric tensors}


\subsection{What are stresses?}\label{ch:stresses}
\textsc{We know what a force is.} Intuitively,
when thinking about fluids or materials, the stress is a force acting on
a surface. We also know that we can add forces, so we can safely assume that
the sum of forces acting on parts of the surface have to add up to the force
acting on the whole surface. Let's define force differential \(\dd{\vec{F}}\) acting
on a small part of the surface \(\dd S\), in a following way
\[
	\dd \vec{F} = \vec{f}\dd{S},
\]
so that \(\vec{f}\) becomes a force per unit surface, so surface force density.

Let us consider a Cartesian coordinate system \((O, x_1, x_2, x_3)\) and
a small tetrahedron with walls \(\dd S_1\), \(\dd S_2\), \(\dd S_3\) and
closed by the wall \(\dd S\) (see \autoref{fig:forcetetrahedron}). There
can be infinite amount of such tetrahedrons, however the numbered walls
\(\dd S_i\) are always laying on the same planes, perpendicular to the
axis of the system. On the other hand, wall \(\dd S\) changes its position
along with the plane its laying on and the surface area.

\begin{marginfigure}
	\centering
  
\begin{tikzpicture}
	\fill [p11, fill opacity = 0.2] (0cm, 0cm) -- (-2cm * 0.75, -1.5cm * 0.75) -- (0cm, 2cm) -- cycle;
	\fill [p12, fill opacity = 0.2] (0cm, 0cm) -- (-2cm * 0.75, -1.5cm * 0.75) -- (2cm, 0cm) -- cycle;
	\fill [p13, fill opacity = 0.2] (0cm, 0cm) -- (2cm, 0cm) -- (0cm, 2cm) -- cycle;
	\fill [p13, fill opacity = 0.2] (0cm, 0cm) -- (2cm, 0cm) -- (0cm, 2cm) -- cycle;

	\draw [p11, -latex, ultra thick]
	(-0.2cm, 0.3cm) -- (-1.7cm, 1cm) node [above] {$\vec{f}_1$};
	\draw [p13, -latex, ultra thick]
	(0.5cm, 0.3cm) -- (2.7cm, 1cm) node [above] {$\vec{f}_3$};
	\draw [p12, -latex, ultra thick]
	(0.3cm, -0.2cm) -- (1.5cm, -1cm) node [above] {$\vec{f}_2$};

	\draw [-latex, thick, p11] (0cm, 0cm) -- (2.75cm, 0cm) node [above] {$x_1$};
	\draw [-latex, thick, p12] (0cm, 0cm) -- (0cm, 3cm) node [above] {$x_2$};
	\draw [-latex, thick, p13] (0cm, 0cm) -- (-2cm, -1.5cm) node [above] {$x_3$};
	\draw [thick] (-2cm * 0.75, -1.5cm * 0.75)  -- (2cm, 0cm) -- (0cm, 2cm) -- cycle;
	% \fill [pattern=north west lines, pattern color=black] (-2cm * 0.75, -1.5cm * 0.75)  -- (2cm, 0cm) -- (0cm, 2cm) -- cycle;

	\draw [p13, thick] (0.5cm, 0.5cm) -- (2cm, 2cm)
	node [pos=1, above] {$\dd S_3$}
	node [draw=p13, circle, fill=p13, inner sep=1pt, pos=0]  {};
	\draw [p11, thick] (-0.2cm, 0.5cm) -- (-1cm, 2cm)
	node [pos=1, above] {$\dd S_1$}
	node [draw=p11, circle, fill=p11, inner sep=1pt, pos=0]  {};
	\draw [p12, ultra thick] (-0.1cm, -0.4cm) -- (-1cm, -1.5cm)
	node [pos=1, below] {$\dd S_2$}
	node [draw=p12, circle, fill=p12, inner sep=1pt, pos=0]  {};
	\draw [black, thick] (0cm, -0.65cm) -- (1cm, -1.5cm)
	node [pos=1, below] {$\dd S$};

	\draw [ultra thick, -latex] (0.2cm, 0.3cm) -- (1.3cm, 2cm)
	node [above] {$\vec{f}$};


\end{tikzpicture}

	\caption{Elementary tetrahedron}
	\label{fig:forcetetrahedron}
\end{marginfigure}

If this volume is in equilibrium, than the forces acting on the walls,
\(\vec{f}_1\), \(\vec{f}_2\), \(\vec{f}_3\) and \(\vec{f}\) respectively,
add up to \(\vec{0}\)
\begin{equation}\label{eq::equilibrium}
  {\color{p11}\vec{f}_1 \dd{S_1}}
  + {\color{p12}\vec{f}_2 \dd{S_2}}
  + {\color{p13}\vec{f}_3 \dd{S_3}}
	+ \vec{f} \dd{S} = \vec{f}_i \dd{S_i} + \vec{f} \dd{S} = \vec{0}.
\end{equation}
Let us divide the equation by \(\dd{S}\)
\[
	\vec{f}_i \dfrac{\dd{S_i}}{\dd{S}} + \vec{f} = \vec{0}.
\]
We do that because we notice that the fraction \(\dd{S} / \dd{S_i}\) is
equal to the cosine of the angle between the normal of \(\dd{S}\) and
the \(i\)-th axis of the coordinate system (believe me, or prove it
if you don't). We can also say that \(\cos(\bm{n}, x_i) = n_i\) and we will
use that later.

In \autoref{eq::equilibrium} we didn't assume anything about the directions
of the forces, however, if the tetrahedron is in equilibrium, than the sum of
\(\vec{f}_i\) has to be opposite to \(\vec{f}\). Hence, we can flip it noting
that if we change direction, the sign changes as well, i.e.\
\(\vec{f}(\vec{n}) = -\vec{f}(-\vec{n})\).
After substituting our \say{revolutionary} findings about the
\(\dd{S} / \dd{S_i}\) ratio equal to the cosine as well, we are left with the following relation
\[
	\vec{f} = \vec{f}_i \dfrac{\dd{S_i}}{\dd{S}}
	= \vec{f}_i\cos(\bm{n}, x_i)
	= \vec{f}_i n_i
\]

Now, considering that each force has three components, which for example for
\(\vec{f}_1\) we will denote as \(f_{11}\), \(f_{12}\) and \(f_{13}\), we can
also write the equation for \(k\)-th component of \(\vec{f}\)
\[
	f_k = n_1 f_{1k} + n_2 f_{2k} +n_3 f_{3k} = n_\alpha f_{\alpha k}.
\]
So, we actually have arrived at something! We have \emph{nine} numbers giving
together the components of \(f_{\alpha k}\), which describe the forces acting
on the surface closing our small tetrahedron. Furthermore, if you shrink the
tetrahedron to a point (we already used differentials of surfaces), what we
wrote above still holds! This means that, the force density acting on a surface
in the fluid depends on those nine numbers and the orientation of the surface.

\begin{marginfigure}
  \begin{tikzpicture}
		\begin{scope}[xshift=1cm]
			\draw [fill, color = black, fill opacity=0.2, thick]
			(5cm, 2.5cm) -- (5.5cm, 2cm)
			-- (5.5cm, 1cm) -- (5cm, 1.5cm) -- cycle ;
			\node at (5cm, 3cm) [above] {normal stress};
			\node at (5cm, 2.60cm) [above]
			{\small \textit{force perpendicular to the surface}};

			\draw [-latex, ultra thick] (4cm, 1.75cm) -- (5.35cm, 1.75cm)
			node [above, pos=0] {$\vb{f}$};

			\node at (5cm, 0.05cm) [above] {shear stress};
			\node at (5cm, -0.35cm) [above]
			{\small \textit{force and surface coplanar}};
			\draw [fill, color = black, fill opacity=0.2, thick]
			(5cm, -0.5cm) -- (6cm, -0.5cm) -- (5.5cm, -1cm)
			-- (4.5cm, -1cm) -- cycle;
			\draw [-latex, ultra thick] (4cm, -0.75cm) -- (5.35cm, -0.75cm)
			node [below, pos=0] {$\vb{f}$};
		\end{scope}
\end{tikzpicture}

  \caption{Schematic visualsation of force generating normal or shear 
  stress depending on the rotation of the surface it acts on.}
  \label{fig:stress_visualisation}
\end{marginfigure}


We will call those numbers stresses. Make a though experiment: imagine a force
acting in a chosen direction, and a surface perpendicular to this force. This
is aligned with the classical definition of pressure, right? Now, rotate this
surface, so that the force vector lies on the surface. Force acting along the
surface can be understood as shear. That's how scissors work! Now, rotate the
surface keeping the normal as an axis. The force is now \say{shearing} the
surface in a different direction, so for one force vector we can have one normal
(perpendicular to the surface)
stress and we observed two shear stresses. With that (hopefully) intuitive
explanation we can say that among the nine numbers, we will have \emph{three}
normal stress components and \emph{six} components related to shear stress.

The only problem is the fact that our notation is cumbersome; we have two \(f\)
symbols. Let's set the magic nine numbers (they will get a name soon, I promise!)
as \(\stress_{ij}\), giving us:
\[
	f_i = n_\alpha \stress_{\alpha i}
\]

\paragraph{Stress tensor}
Okay, so as I said, the nine-number-thingy needs a name. You saw the paragraph
title, you might already now what it is. That said, this is written in style of
a lecture, so please make a dramatic pause in your brain before reading the end of the
next sentence. We will call \(\stress_{ij}\) the \emph{stress tensor}.

But what is a tensor, you might ask? My colleague said at some point\footnote{In Dresden, on the 1st of June, during breakfast, around 9:30 AM.} that a tensor
is just a \say{glorified vector}. This is not really true, but it is a useful starting point for a discussion.
In fact, some vectors
\emph{are} tensors. Generally, a tensor is a \say{geometrical thing}, that is,
it describes some effect independent on the coordinate system. Velocity vector
is a tensor or order 1, as it describes that an object moves in a specific direction
with a specific velocity\footnote{Not every vector is a tensor, e.g.\ the
	position vector is always dependant on the chosen coordinate system.}. If we
change the basis (coordinate system), the representation, that is the components
of the vector will change, but the vector itself still points in the same
direction. \emph{It is a geometric concept.} Similarly to our velocity example,
stress tensor also describe a distribution of forces inside the material. In
nature there are no principal, favoured directions, so the object describing it
should be a tensor.

So, there is some geometric transformation
that tells us how to change the components of a tensors, and tensors
are objects that obey this transformation.
This paraphrase of a mathematical mathematical definition is
frankly quite unhelpful\footnote{% 
For interested here is slightly paraphrased version of a definition,
again stolen from Wikipedia: A tensor of type \((p,q)\) is an assignment
of a multidimensional array
\[
	T^{i_1\dots i_p}_{j_{1}\dots j_{q}}[\mathbf{f}]
\]
to each basis \(\vec{f} = (\vec{e}_1, \dots, \vec{e}_n)\) of an \(n\)-dimensional
vector space such that, if we apply the change of basis
\[
	\mathbf{f}\mapsto \mathbf{f}\cdot R = \left( \mathbf{e}_i R^i_1, \dots, \mathbf{e}_i R^i_n \right)
\]
then the multidimensional array obeys the transformation law
\[
	T^{i'_1\dots i'_p}_{j'_1\dots j'_q}[\mathbf{f} \cdot R] = \left(R^{-1}\right)^{i'_1}_{i_1} \cdots \left(R^{-1}\right)^{i'_p}_{i_p}
	T^{i_1, \ldots, i_p}_{j_1, \ldots, j_q}[\mathbf{f}]
	R^{j_1}_{j'_1}\cdots R^{j_q}_{j'_q} .
\]
}.
So we will cheat a little
and describe a tensor with as much verbosity as is necessary for our use case.

Let's define some operations first and show them on the example of the stress
tensor, which we will represent here as
\[
	\stressT = \stress_{ij}\hat{\vec{e}}_i \hat{\vec{e}}_j.
\]
We used the versors \(\hat{\vec{e}}_i\) defining our chosen coordinate system.
They are multiplied in a weird way, its neither scalar nor vector product. This
operation is also sometimes written with
the \(\otimes\) symbol (\(\hat{\vec{e}}_i \otimes \hat{\vec{e}}_j\)) and
is called tensor or outer product.

The scalar product \(\vec{n}\dotproduct\stressT\) can be written easily
with index notation. We know that \(\vec{n} = n_k \hat{\vec{e}}_k\), so
building on that we can write
\[
	\vec{n}\dotproduct\stressT
	= n_k \hat{\vec{e}}_k \dotproduct \qty(\stress_{ij}\hat{\vec{e}}_i \hat{\vec{e}}_j)
	= n_k \stress_{ij} \hat{\vec{e}}_k \dotproduct \hat{\vec{e}}_i \hat{\vec{e}}_j
	= n_k \stress_{ij} \qty(\hat{\vec{e}}_k \dotproduct \hat{\vec{e}}_i) \hat{\vec{e}}_j.
\]
We can think a little about the term \(\hat{\vec{e}}_k \dotproduct \hat{\vec{e}}_i\).
It is a dot product of two versors. It can be either equal to 1 if they are
the same (\(k=i\)) or 0 if they are different. There is a term for that, called
\emph{Kronecker delta}, defined as
\[
	\delta_{ij} = \begin{cases}
		0 & \text{if } i \neq j, \\
		1 & \text{if } i=j.\end{cases}
\]
So we can write that our product is equal to
\(n_k \stress_{ij}\delta_{ki} \hat{\vec{e}}_j \). It does not help us yet, but if
we said that \(k\) has to be equal to \(i\), otherwise the terms in the sum
will be 0, we might as well write \(i\) instead of \(k\)\footnote{If you are
	confused try it out on paper, write every term, eliminate the ones equal to 0
	and collapse the term to the indexed expression.}, i.e.\
\[
	\vec{n}\dotproduct\stressT = n_i \stress_{ij} \hat{\vec{e}}_j = f_j \hat{\vec{e}}_j.
\]
In the above, we eliminated \(\delta_{ii}\) as it is always equal to 1. In fact,
\(\delta_{ij}\) is sometimes called an \say{index renamer} as it renames index
\(j\) to \(i\) or \textit{vice versa}.

Okay, so what have we learned? The dot product of a tensor with a vector gives
us a vector. Will multiplication from the left, which we performed above, be
different from the one from the right, that is \(\stressT\dotproduct\vec{n}\)?
Yes\footnote{I am to lazy to write it down, but it might be a good training
	for you.}. That said, it would be the same if \(\stress_{ij} = \stress_{ji}\).

So a lot of hints have been dropped. The tensor behaves a little bit like a
matrix. In fact the matrix is often used to \emph{represent} a tensor in a
chosen coordinate system, but the tensor itself, like a velocity vector, is
independent from the chosen coordinate system\footnote{Stress tensor has rank 2; we
	used two versors, so called diad, to define it.}. This matrix can be multiplied with a vector to give
another vector. If the matrix is symmetric, product from both sides will return vectors
with the same components. Kronecker delta is also a tensor itself, represented by
a unit matrix \(\tnsr{I}\).

\subsection{Strain-rate (deformation-rate) tensor}
It was mentioned, that we are interested in a tensor defining a rate of
deformation, or what is usually called \emph{strain-rate tensor} in
literature. We should derive it. Let's start with a series expansion of
velocity field around coordinate \(\vb{r}\):
\[
	v_i(t, \vb{r} + \dd{\vb{x}}) = v_i(t, \vb{r}) + \pdv{v_i}{x_k}\dd{x_k}
	+ \dots
\]
If \(\dd{x}\rightarrow\vb{0}\), we disregard the higher order terms and
rewrite the above as
\[
	v_i(t, \vb{r} + \dd{\vb{x}}) = v_i(t, \vb{r})
	+ \frac{1}{2}\qty(\pdv{v_i}{x_k} + \pdv{v_k}{x_i})\dd{x_k}
	+ \frac{1}{2}\qty(\pdv{v_i}{x_k} - \pdv{v_k}{x_i})\dd{x_k}.
\]
The first term on the right describes simple translation of the fluid parcel.
The third (similar to the expression for the curl of velocity field) defines
rotation. Both of these can be applied to the rigid, unchanging in shape,
fluid parcel. We are left with the second one, describing the change of
velocity due to \say{non-rigid} strain or deformation of the fluid volume.
We can write this change of velocity mathematically as
\begin{equation}\label{eq::strain_rate_tensor}
	\dd{v^\mathrm{strain}_k}
	= \frac{1}{2}\qty(\pdv{v_i}{x_k} + \pdv{v_k}{x_i})\dd{x_k}
	= \dot{D}_{ik}\dd{x_k}
\end{equation}
The tensor \(\dot{\vb{D}}\) is the strain-rate tensor and very importantly
it is \emph{symmetric}!



\section{Road towards momentum equation}

\subsection{Force acting on a surface}
Armed with the stress tensor we can now compute the force acting on any
closed surface:
\[
	\vb{F} = \oint\limits_S \vb{f} \dd{S}
	= \oint\limits_S \vb{n}\bm{\tau}
	= \hat{\vb{e_i}}\oint\limits_S n_j \tau_{ji}\dd{S}
\]
Now we can use the Greens theorem\footnote{In polish it is called
	Green-Gauss-Ostrogracki (GGO) theorem.} to convert the surface integral
to a volume one. Or simply, when mathematicians aren't looking, it is
a theorem about changing \(S\) for \(\Omega\) and \(n_j\) for
\(\partial/\partial x_j\). Naturally, \(S\) is the surface of \(\Omega\).
Let's do the deed
\[
	\vb{F}
	= \hat{\vb{e_i}}\oint\limits_S n_j \tau_{ji}\dd{S}
	= \hat{\vb{e_i}}\int\limits_\Omega \pdv{\tau_{ji}}{x_j}\dd{\Omega}
	= \int\limits_\Omega \div\bm{\tau} \dd{\Omega}
\]

\subsection{Body force}
We have written an expression for surface forces. What about the forces
acting from the \emph{inside}, like gravity? Here we can say
\[
	\vb{F} = \int\limits_\Omega \vb{f} \dd{m}
\]
Pardon my messy notation, but now \(\vb{f}\) is the \emph{volume} density
of force. We also wrote \(\dd{m}\) which is the mass differential. It is
simply given as \(\rho\dd{\Omega}\), where the first in the product is the
density of the material and the second is the volume differential. So
\[
	\vb{F} = \int\limits_\Omega \rho\vb{f} \dd{\Omega}
\]

\subsection{Derivation starts}
We finally start deriving the momentum equation of a continuous medium.
We start from basic principles
\[
	\dv{t}\text{momentum} = {\text{force}}
\]
We first need to handle the momentum part. We divide the fluid into small,
moving parts. All the parts give the sum of the whole momentum, so we write
it as an integral
\[
	\text{momentum} = \int\limits_\Omega \dd{ (\text{momentum})}.
\]
We will handle the forces in a second. The differential of linear momentum
can be simply expressed as \(\vb{v}\dd{m} = \rho\vb{v} \dd{\Omega}\), where
\(\rho\) is the density. We already wrote the expression for the forces so
if we assemble everything we get this
\begin{equation}\label{eq::mom_part}
	\dv{t}\int\limits_\Omega \rho\vb{v} \dd{\Omega} =
	\int\limits_\Omega  \div\bm{\tau} + \rho\vb{f} \dd{\Omega}
\end{equation}
This is a simple equation, but it does not give us much insight. On the other
hand it describes the movement of \emph{everything}. One important thing to
note is that the volume \(\Omega\) is completely arbitrary. \emph{We can
	chose it as we see fit.} We will use this property later to obtain
differential versions of the equations.

\subsection{Reynolds transport theorem}
We are now standing in front of a difficult problem. We have to
differentiate an integral, which usually is not a sophisticated
issue. However, in our case, the integration domain \(\Omega\)
is not a rigid body. Instead, it deforms and changes shape as the time progresses
and the fluid flows, i.e.\ \(\Omega = \Omega(t)\). In short it depends on
\emph{the variable with respect to witch we differentiate}.
Fortunately, we have a tool for that: Reynolds Transport Theorem.

\begin{theorem}[Reynolds transport theorem, RTT]
	Given:
	\begin{enumerate}[(i)]
		\item a material volume \(\Omega\), whose shape and position are varying with time,
		      i.e.\ \(\Omega = \Omega(t)\);
		\item an extensive variable \(\Phi = \Phi(t)\) with its intensive counterpart
		      \(\phi = \phi(t, \vb{x})\)  defined inside the material volume.
	\end{enumerate}
	The total derivative of \(\Phi\) with respect to time can be formulated as:
	\begin{equation}
		\dv{\Phi}{t} = \dv{t} \int\limits_{\Omega(t)} \phi \dd{\Omega}
		= \int\limits_{\Omega(t)} \dfrac{D\phi}{Dt} + \phi \div{\vb{v}} \dd{\Omega}
	\end{equation}
\end{theorem}

As can be seen above, the theorem connects the time derivative of an integral
similar to ours with the integral of derivatives. The expression \(D\phi/Dt\)
is a substantial derivative, defined as
\[
	\dfrac{D\phi}{Dt} = \pdv{\phi}{t} + v_i\pdv{\phi}{x_i}
\]
To showcase how it works, we will use RTT to derive the continuity equation.

\subsection{Detour for conservation of mass}
The mas is conserved. This can be written as
\[
	\dv{t}\int\limits_\Omega \dd{m} =
	\dv{t}\int\limits_\Omega \rho\dd{\Omega} = 0
\]
We see the time derivative of an integral. The alarm bells have sounded.
The war horns are sounding the music of battle. The ghost of all scientists
pushing the boundaries of fluid dynamics, since human first managed to produced
moonshine by heating up a piece of copper pipe, are eagerly awaiting our next
move. We were preparing for this moment
for the last 10 minutes while reading last paragraph about RTT.
\emph{We what to do.}
\[
	\dv{t}\int\limits_\Omega \rho\dd{\Omega} =
	\int\limits_\Omega  \dfrac{D\rho}{Dt} + \rho\div{\vb{v}}\dd{\Omega}
	=  0
\]
The expression under the integral can be further expanded to
\[
	\int\limits_\Omega  \pdv{\rho}{t} + (\vb{v}\dotproduct\grad)\rho
	+ \rho\div{\vb{v}}\dd{\Omega}
\]
and if we note that
\[
	(\vb{v}\dotproduct\grad)\rho + \rho\div{\vb{v}} =
	v_i \pdv{\rho}{x_i} + \rho\pdv{v_i}{x_i} = \pdv{\rho v_i}{x_i}
	= \div{(\rho \vb{v})}
\]
we arrive at (almost) final result
\[
	\int\limits_\Omega  \pdv{\rho}{t} + \div{(\rho \vb{v})}\dd{\Omega} = 0
\]
Now, we will use an observation about the integration region from the
previous paragraphs, that is, it can be chosen arbitrarily. We haven't
made any assumptions about it, so intuitively it shouldn't matter.

Let's think about for a second. For the sake of shortening the notation
lets denote \({\partial\rho}/{\partial t} + \div{(\rho \vb{v})}\) as \(X\). For a
completely arbitrary region \(\Omega\), the integral \(\int_\Omega X \dd{\Omega}\)
is always equal to zero. This we know from the derived mass conservation equation.
Of coarse, it will be true when \(X=0\) everywhere. Still, can it be the case that
that \(X\neq0\) but the integral is equal to 0?

For that to be true, the region \(\Omega\) should be composed of \(\Omega^+\),
a part in which \(X>0\), and \(\Omega^-\) where \(X<0\). Together, positive
contribution of \(\Omega^+\) and negative from \(\Omega^-\) give null value
of the integral.

But wait! We said, that the \(\Omega\) can be chosen arbitrarily. So what
if we chose \(\Omega = \Omega^+\)? We know that the integral has to be zero,
but that is impossible! We established that in this region \(X\) is strictly
positive and, therefore, the integral has to be positive as well. We have
reach a result confiction with our assumptions: \emph{this option is
	impossible.} We can safely say, that the only way for the integral to
be null for each region, is for the integrated expression to equal to 0
in each point in the region. So we can simply write
\[
	\pdv{\rho}{t} + \div{(\rho \vb{v})} = 0
\]
which competes the derivation of mass conservation law.


\subsection{Back to momentum}
We use RTT agian, this time on \autoref{eq::mom_part}. We get:
\[
	\int\limits_\Omega \pdv{\rho \vb{v}}{t}
	+ \div{\qty(\rho\vb{v}\vb{v} )} \dd{\Omega} =
	\int\limits_\Omega \div\bm{\tau} + \rho\vb{f} \dd{\Omega}
\]
The product of velocity \(\vb{v}\vb{v}\) is a tensor. We can also write
it as \(v_i v_j\). We note that we can actually accumulate the terms into
one integral equal to zero. We saw something like that before, right?
We \emph{again} know what to do! After dropping the integration we are
left with:
\begin{equation}\label{eq::momentum_general_form}
	\pdv{\rho \vb{v}}{t}
	+ \div{\qty(\rho\vb{v}\vb{v} )}  =
	\div\bm{\tau} + \rho\vb{f}
\end{equation}
which is a general, differential form of the momentum conservation law.
However, in order to move forward we have to actually handle the issue
mentioned in the very beginning: defining a description for the stress
tensor of our fluid, or the \emph{constitutive relation}.

\section{Fluid constitutive relations}

We should now try to derive the formulas for \(\bm{\tau}\) so we can plug
them into the momentum equation, be done with this derivation and go home.
Let's start with a simple model of an \emph{inviscid} fluid. If the fluid
has no viscosity, than pressure should be the only thing contributing to
the stress tensor. \emph{For the sake of later discussion, please note that
	this is the same situation as in the non-moving viscous fluid.} We remember
Pascal's law
\begin{quote}
	\say{Any two points at the same elevation in a continuous mass of the same
		static fluid will be at the same pressure}\footnote{%
		White, F.M. (2016) Fluid mechanics. Eighth edition. New York,
		NY: McGraw-Hill Education.}
\end{quote}
here stated assuming hydrostatic distribution of pressure. If we don't have
any favoured direction it can be rephrased as
\begin{quote}
	\say{The pressure applied to any part of the enclosed liquid will be
		transmitted equally in all directions through the liquid}\footnote{%
		Wikipedia (Accessed 01.06.2023), Pascal's law}
\end{quote}
We can than represent the force density acting at \(\dd{S}\) as
\[
	\vb{f} = -\vb{n}p,
\]
where \(p\) represents pressure. We assume that the normal vector \(\vb{n}\) is
directed outside of a fluid parcel that we are considering. The minus in front
of the expression, is there due to the fact that the surface exerts the pressure
inwards. If we note that \(\vb{f} = \vb{n}\dotproduct\bm{\tau}\) we can now say
\[
	\vb{f} = \vb{n}\dotproduct\bm{\tau} = -\vb{n}p = \vb{n}\dotproduct \qty( -p\vb{I} )
	\rightarrow \bm{\tau} = -p \vb{I}
\]
or in the index notation \(\tau_{ij} = -p \delta_{ij}\).


\subsection{Including strain-rate}
Let's generalise. As we mentioned before, we are looking for
relation of the form \(\bm{\tau} = g(\dot{\vb{D}})\). We would like to do the
power expansion of \(g\)\footnote{Power series around 0 looks like that:
	\(k(x) = k(0) + ax + bx^2 + \dots\)}. We will have a constant part (independent
of the function argument), linear, quadratic and so on. In our case, the
constant part can be conveniently expressed as
\(g(\dot{\vb{D}})_{\dot{\vb{D}} = \vb{0}}\). This situation is exactly like the
one before, we have no deformation, so the fluid is probably not moving. So
we can write:
\[
	\bm{\tau} = -p\vb{I} + \text{linear term w.r.t. } \dot{\vb{D}}
	+ \text{quadratic term w.r.t. } \dot{\vb{D}} + \dots
\]
We have the constant term and now we need to determine the other ones.

\subsection{Eigenvalue problem for a tensor}
For every tensor, it can \say{act} on a vector to produce another vector, as
in \(M_{\alpha\beta}a_\beta = b_\alpha\) or \(\vb{M}\vb{a} = \vb{b}\). If we
have selected a special vector \(\vb{a}\), the resultant vector \(\vb{b}\)
might be collinear with \(\vb{a}\). This property is a geometric fact,
independent of the chosen coordinate system. So if we set this special
vector as \(\vb{q}\) and following is true
\[
	\vb{M}\vb{q} = \lambda\vb{q} \quad \text{or} \quad
	\qty(\vb{M} - \lambda\vb{I})\vb{q} = \vb{0},
\]
than both \(\lambda\), called an \emph{eigenvalue}, and \(\vb{q}\), the
\emph{eigenvector} do not depend on the coordinate system.
For \(\vb{q}\) not to be a trivial \(\vb{0}\), following determinant has to
vanish
\[
	\det(\vb{M} - \lambda\vb{I}) = 0.
\]
If we actually spell the determinant out we will get the third order
equation for \(\lambda\):
\begin{equation}\label{eq::characterstic_equation}
	-\lambda^3 + I_1 \qty(\vb{M})\lambda^2 - I_2\qty(\vb{M})\lambda
	+ I_3\qty(\vb{M}) = 0.
\end{equation}
It is called \emph{characteristic equation} of a tensor (or a matrix). The
scalars \(I_1\), \(I_2\) and \(I_3\) depend on \(\vb{M}\). We established
that eigenvectors do not depend on the coordinate system, they \say{zero}
\autoref{eq::characterstic_equation}, so the coefficients of the equation,
that is \(I_1\), \(I_2\) and \(I_3\), also have to be independent from the
chosen system. They are called \emph{invariants} and are given by the
following relations\footnote{Shamelessly stolen from Wikipedia (accessed
	01.06.2023), Invariants of tensors}:
\begin{align*}
	I_1 & = \mathrm{tr}(\mathbf{M}) = M_{11}+M_{22}+M_{33} = \lambda_1+\lambda_2+\lambda_3                       \\
	I_2 & = \frac{1}{2} \left( (\mathrm{tr}(\mathbf{M}))^2-\mathrm{tr} \left( \mathbf{M}^2 \right) \right)       \\
	    & = M_{11}M_{22}+M_{22}M_{33}+M_{11}M_{33}-M_{12}M_{21}-M_{23}M_{32}-M_{13}M_{31}                        \\
	    & = \lambda_1 \lambda_2 + \lambda_1 \lambda_3 + \lambda_2 \lambda_3                                      \\
	I_3 & = \det (\mathbf{M}) = -M_{13} M_{22} M_{31} + M_{12} M_{23} M_{31} + M_{13} M_{21} M_{32}              \\
	    & - M_{11} M_{23} M_{32} - M_{12} M_{21} M_{33} +   M_{11} M_{22} M_{33} = \lambda_1 \lambda_2 \lambda_3
\end{align*}
Naturally, remembering those formulas is not useful, but it is good to
remember that they exist. We will now introduce a theorem that will do
a lot of work for us.
\begin{theorem}[Cayley–Hamilton theorem]
	Every square matrix over a commutative ring (such as the real or
	complex numbers or the integers) satisfies its own characteristic equation.
\end{theorem}
\noindent
In practice, this means that the following holds
\begin{equation*}
	-\vb{M}^3 + I_1 \qty(\vb{M})\vb{M}^2 - I_2\qty(\vb{M})\vb{M}
	+ I_3\qty(\vb{M})\vb{I} = 0.
\end{equation*}
We essentially substituted \(\vb{M}\) into \autoref{eq::characterstic_equation}.
What can now rearrange and obtain closed expression for \(\vb{M}^3\)
\begin{equation}\label{eq::third_power}
	\vb{M}^3  = - I_1 \qty(\vb{M})\vb{M}^2 + I_2\qty(\vb{M})\vb{M}
	- I_3\qty(\vb{M})\vb{I}
\end{equation}
and \(\vb{M}^4\) and so on.

\begin{fullwidth}
\bibliography{refs}
\bibliographystyle{plainnat}
\end{fullwidth}


\end{document}
